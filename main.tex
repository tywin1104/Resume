 %%%%%%%%%%%%%%%%%
% This is an sample CV template created using altacv.cls
% (v1.1.4, 27 July 2018) written by LianTze Lim (liantze@gmail.com). Now compiles with pdfLaTeX, XeLaTeX and LuaLaTeX.
% 
%% It may be distributed and/or modified under the
%% conditions of the LaTeX Project Public License, either version 1.3
%% of this license or (at your option) any later version.
%% The latest version of this license is in
%%    http://www.latex-project.org/lppl.txt
%% and version 1.3 or later is part of all distributions of LaTeX
%% version 2003/12/01 or later. 
%%%%%%%%%%%%%%%%

%% If you need to pass whatever options to xcolor
\PassOptionsToPackage{dvipsnames}{xcolor}


\documentclass[10pt,a4paper]{altacv}
%% AltaCV uses the fontawesome and academicon fonts
%% and packages. 
%% See texdoc.net/pkg/fontawecome and http://texdoc.net/pkg/academicons for full list of symbols.
%% 
%% Compile with LuaLaTeX for best results. If you
%% want to use XeLaTeX, you may need to install
%% Academicons.ttf in your operating system's font 
%% folder.


% Change the page layout if you need to
\geometry{left=1cm,right=9cm,marginparwidth=6.8cm,marginparsep=1.2cm,top=1.25cm,bottom=1.25cm,footskip=2\baselineskip}

% Change the font if you want to.

% If using pdflatex:
\usepackage[T1]{fontenc}
\usepackage[utf8]{inputenc}
\usepackage[default]{lato}

% If using xelatex or lualatex:
% \setmainfont{Lato}

% Change the colours if you want to
\definecolor{Mulberry}{HTML}{72243D}
\definecolor{SlateGrey}{HTML}{2E2E2E}
\definecolor{LightGrey}{HTML}{000000}
\colorlet{heading}{Sepia}
\colorlet{accent}{Mulberry}
\colorlet{emphasis}{SlateGrey}
\colorlet{body}{LightGrey}
\usepackage{hyperref}
\hypersetup{
    colorlinks,
    citecolor=black,
    filecolor=black,
    linkcolor=black,
    urlcolor=black
}
% Change the bullets for itemize and rating marker
% for \cvskill if you want to
\renewcommand{\itemmarker}{{\small\textbullet}}
\renewcommand{\ratingmarker}{\faCircle}
%% sample.bib contains your publications
\addbibresource{sample.bib}

\usepackage[colorlinks]{hyperref}

\begin{document}

\name{TIANYI \space ZHANG}
\tagline{}
% \photo{2.8cm}{Globe_High}
\personalinfo{%
  % Not all of these are required!
  % You can add your own with \printinfo{symbol}{detail}
  \email{tiaven1104@gmail.com}
  \phone{647-936-2123}
%   \mailaddress{Address, Street, 00000 County}
%   \location{Location, COUNTRY}
  \homepage{https://tywinzhang.com}
%   \twitter{@twitterhandle} 
\\
  \linkedin{\href{https://www.linkedin.com/in/tywinzhang/}{https://www.linkedin.com/in/tywinzhang/}} 
  \github{\href{https://github.com/tywin1104}{https://github.com/tywin1104}}
  %% You MUST add the academicons option to \documentclass, then compile with LuaLaTeX or XeLaTeX, if you want to use \orcid or other academicons commands.
%   \orcid{orcid.org/0000-0000-0000-0000}
}

%% Make the header extend all the way to the right, if you want. 
\begin{fullwidth}
\makecvheader
\end{fullwidth}

%% Depending on your tastes, you may want to make fonts of itemize environments slightly smaller
% \AtBeginEnvironment{itemize}{\small}


%% Provide the file name containing the sidebar contents as an optional parameter to \cvsection.
%% You can always just use \marginpar{...} if you do
%% not need to align the top of the contents to any
%% \cvsection title in the "main" bar.
\cvsection[page1sidebar]{Working Experience}

\cvevent{Backend Developer Intern}{IBM Canada - Db2 on Cloud}{May 2019 - Present}{Markham, Ontario, Canada}
 \begin{itemize} 
  \item Lead contributor of multiple \textbf{control plane} components for a declarative \textbf{controller-based} \textbf{cloud-native} platform running elastic stateful Db2 services with \textbf{High Availability} on\textbf{ Kubernetes}
  \item Implemented the billing microservice with \textbf{Go}, \textbf{Kubernetes controllers}, and \textbf{Postgresql} to collect and submit high-watermark customer usage for multiple metrics
  \item Developed the \textbf{gRPC} microservice for dynamically dispatching operations onto customer database instances in \textbf{Kubernetes} clusters with \textbf{Go} \\
%   \item Integrated our service with IBM Cloud Platform via \textbf{service broker} API written in \textbf{Ruby on Rails}, enabled \textbf{IAM} authentication and authorization for the service APIs.
 \item Built, tested and delivered an automated orchestration pipeline for provisioning production Db2 on cloud \textbf{High Availability} instances on AWS with \textbf{Ansible} \\
%  \item Revamped tools to analyze Db2 application snapshots, achieving \textbf{concurrent processing} and reduced processing time from \textbf{3 mins} to \textbf{2 secs} using \textbf{Go}
%   \item Delivered a web-based tool for Db2 diagnostic logs analysis using \textbf{Python, Nginx, and Docker}, automating \textbf{3} workloads and significantly enhanced productivity for the team	
  
 \end{itemize}
 
 \divider
 
\cvevent{Teaching Assistant}{Department of Computing and Software, McMaster University}{Sep 2018 -- May 2019}{Hamilton, Ontario, Canada}
 \begin{itemize} 
 \item Led weekly tutorials and provided academic assistance to \textbf{100+} students in two core CS courses: Intro to Computational Thinking (\textbf{Haskell}), Intro to Programming (\textbf{Python})
 \item Significantly improved making efficiency by automating weekly grading tasks for the D2L system using 
\textbf{Selenium web automation} and \textbf{data scraping}
 
 
 \end{itemize}
\divider

\cvevent{Full Stack Developer Intern}{Royal Bank of Canada
}{May 2018 - Sep 2018}{Toronto, Ontario, Canada}
\begin{itemize}
  \item Developed an AI-powered business intelligence system for advisors to better engage with clients using \textbf{Python Flask} and \textbf{Angular}
  \item Integrated \textbf{business data} from various sources and generated daily industry reports by automating the workflow of \textbf{5} different tasks
%   \item Designed and implemented scalable \textbf{MS SQL} database model and multiple \textbf{REST API }endpoints
  \item Implemented caching layer to decrease network calls and improved runtime performance by \textbf{40\%}

 \end{itemize} 

% \divider

% \cvevent{Application Developer}{RedReach}{Nov 2017 - May 2018}{Hamilton, Ontario, Canada}
%  \begin{itemize} 
%  \item Built \textbf{prototype} and designed \textbf{user flow} for our startup project that focused on local youth employment
%   \item Developed exceptional \textbf{communication} and \textbf{analytical skills} while conducting user feedback interviews and doing \textbf{business analysis}\\
%  \end{itemize}


\cvsection{Skills}
\cvtag{Flask} \cvtag{Node.js}   \cvtag{HTML}  \cvtag{CSS} \cvtag{Angular} \cvtag{React} \\
 \cvtag{RestAPI} \cvtag{gRPC} \cvtag{SQL}\cvtag{MongoDB} \cvtag{Jenkins}   
\divider\smallskip
\cvtag{Docker} \cvtag{Kubernetes} \cvtag{Helm} \cvtag{AWS}  \cvtag{Git}
\cvtag{Linux} \cvtag{Ansible}


\cvsection{Programming Languages}
\cvskill{Python, Go}{5}
\vspace{2.3mm}
\cvskill{Javascript}{4}
\vspace{2.3mm}
\cvskill{Java, Bash}{3}

\end{document}
